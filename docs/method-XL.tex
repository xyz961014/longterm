%%%%%%%%%%%%%%%%%%%%%%%%%%%%%%%%%%%%%%%%%
% Journal Article
% LaTeX Template
% Version 1.4 (15/5/16)
%
% This template has been downloaded from:
% http://www.LaTeXTemplates.com
%
% Original author:
% Frits Wenneker (http://www.howtotex.com) with extensive modifications by
% Vel (vel@LaTeXTemplates.com)
%
% License:
% CC BY-NC-SA 3.0 (http://creativecommons.org/licenses/by-nc-sa/3.0/)
%
%%%%%%%%%%%%%%%%%%%%%%%%%%%%%%%%%%%%%%%%%

%----------------------------------------------------------------------------------------
%	PACKAGES AND OTHER DOCUMENT CONFIGURATIONS
%----------------------------------------------------------------------------------------

\documentclass{article}

\usepackage{xeCJK}
\usepackage{CJK}         % CJK 中文支持
\usepackage{fancyhdr}
\usepackage{ctex}
\usepackage{amsmath,amsfonts,amssymb,graphicx}    % EPS 图片支持
\usepackage{subfigure}   % 使用子图形
\usepackage{indentfirst} % 中文段落首行缩进
\usepackage{bm}          % 公式中的粗体字符(用命令\boldsymbol)
\usepackage{multicol}    % 正文双栏
\usepackage[sc]{mathpazo} % Use the Palatino font
\usepackage[T1]{fontenc} % Use 8-bit encoding that has 256 glyphs
\linespread{1.05} % Line spacing - Palatino needs more space between lines
\usepackage{microtype} % Slightly tweak font spacing for aesthetics

\usepackage[english]{babel} % Language hyphenation and typographical rules

\usepackage[hmarginratio=1:1,top=32mm,columnsep=20pt]{geometry} % Document margins
\usepackage[hang, small,labelfont=bf,up,textfont=it,up]{caption} % Custom captions under/above floats in tables or figures
\usepackage{booktabs} % Horizontal rules in tables



\usepackage{enumitem} % Customized lists
\setlist[itemize]{noitemsep} % Make itemize lists more compact



\usepackage{fancyhdr} % Headers and footers
\pagestyle{fancy} % All pages have headers and footers
\fancyhead{} % Blank out the default header
\fancyfoot{} % Blank out the default footer
\fancyhead[C]{Cache-based Recurrent Transformer Network} % Custom header text
\fancyfoot[RO,LE]{\thepage} % Custom footer text

\usepackage{titling} % Customizing the title section

\usepackage{hyperref} % For hyperlinks in the PDF

%----------------------------------------------------------------------------------------
%	TITLE SECTION
%----------------------------------------------------------------------------------------

\setlength{\droptitle}{-4\baselineskip} % Move the title up

\pretitle{\begin{center}\Huge\bfseries} % Article title formatting
\posttitle{\end{center}} % Article title closing formatting
\title{Cache-based Recurrent Transformer Network} % Article title
\author{%
% Your institution address
%\and % Uncomment if 2 authors are required, duplicate these 4 lines if more
%\textsc{Jane Smith}\thanks{Corresponding author} \\[1ex] % Second author's name
%\normalsize University of Utah \\ % Second author's institution
%\normalsize \href{mailto:jane@smith.com}{jane@smith.com} % Second author's email address
}
\date{} % Leave empty to omit a date

%----------------------------------------------------------------------------------------

\begin{document}

% Print the title
\maketitle

%----------------------------------------------------------------------------------------
%	ARTICLE CONTENTS
\section*{概览}
	对于时间步t,输入片段(向量)为$X_t \in R^{L\times d}$,Memory为$\mathcal{M} = \{K_i, V_i\}_{i=1}^N$,网络的更新步骤大致分为
	\subsection*{查询}
		查询应该符合如下原则:
		\begin{itemize}
			\item 对于片段中的每个词,应有一个查询结果。
			\item 对于每个词的查询,这个词之后的词应该每mask住。
			\item 对于片段的第一个词,后面的词都是看不见的,查询时信息太少,应该拼接上一个片段$X_{t-1}$参与查询
			\item 从大小为$N$的缓存中查询中$k$个片段
		\end{itemize}
		
		\begin{equation}
			\{\alpha_i, Z_i\}_k^{(j)} = Query(X_{t-1}, Masked_j(X_t), \mathcal{M})
		\end{equation}
		其中$Z_i$为需要进行回忆(以Transformer Memory的方式拼接进当前片段)的k个片段,$\alpha_i$为其对应的权重,$j$表示计算第j个词对应的查询结果,$j = 1,2,\cdots, L$\\
		$\mathcal{M}$是Memory,存储N个Key-Value对,Value即为片段,对应的Key为该片段的一个意义向量。
	\subsection*{更新hidden state}
		使用Transformer网络更新m层隐状态
		\begin{equation}
			h_{t,j}^{1:m} = Transformer(\alpha_{1:k}^{(j)},Z_{1:k}^{(j)}, X_t)
		\end{equation}
	\subsection*{更新Memory}
		\begin{equation}
			\mathcal{M} = renew(h_t^{1:m}, \mathcal{M})
		\end{equation}
\section{查询}
		采取键值对匹配的方法,$Q_t^{(j)}$是对应的查询向量,计算方法列下,$Keys$是缓存中的键向量。
		$$\{\alpha_i, Z_i\}_k^{(j)} = topk(softmax(Q_t^{(j)} \cdot Keys^T))$$
		以下列出查询向量$Q_t^{(j)}$的计算方法:
		\begin{itemize}
			\item \textbf{1.1} fixed\_length 定长方法: $Q_t^{(j)}$包含包括第j个词在内的第j个词之前的L个词(事实上已经取到$X_{t-1}$中,此处仍用$X_t$表示)
			\begin{itemize}
				\item \textbf{1.1.1} fixed\_length\_1 直接使用底层的连续L个词,使用encoder进行计算
					$$Q_t^{(j)} = (X_{t,j-L+1}, \cdots, X_{t,j-1}, X_{t,j})$$ 
					$$Q_t^{(j)} = summary(Transformer(Q_t^{(j)}))$$
				\item \textbf{1.1.2} fixed\_length\_2 将j个词作为当前片段,上一片段中的后L-j个词的表示作为memory
				$$Q_t^{(j)} = Transformer(X_{t-1}[j+1:L], X_t[1:j])$$ 
					$$Q_t^{(j)} = summary(Q_t^{(j)})$$
			\end{itemize}
			
			 其中$summary$函数与之后更新Memory时使用的相同,$Transformer$共享模型参数

		\item \textbf{1.2} 倍长方法:将$X_{t-1}$的L个词拼接入当前输入,使用2L个词进行计算
		$$Q_t^{(j)} = concat(X_{t-1}, Masked_j(X_t))$$
		$$Q_t^{(j)} = Transformer(Q_t^{(j)})$$
		注意到此时$Q_t^{(j)}$为2L个词的表示,需要将其变成L个词的表示,才能使用$summary$函数
		\begin{itemize}
			\item \textbf{1.2.1} last\_l 截取后L个
				$$Q_t^{(j)} = summary(Q_t^{(j)}[L+1:2L])$$
			\item \textbf{1.2.2} middel\_l 截取中间L个
				$$Q_t^{(j)} = summary(Q_t^{(j)}[j+1:j+L])$$
			\item \textbf{1.2.3} linear 线性变换法
				$$Q_t^{(j)} = summary(Linear(Q_t^{(j)}))$$
		\end{itemize}
			
			其中$summary$函数与之后更新Memory时使用的相同, $Transformer$共享模型参数
	\end{itemize}

\section{更新hidden state}
	\begin{itemize}
		\item \textbf{2.1} standard:采用Transformer-XL的方法
			$$h_{t,j}^{1:m} = Transformer(\alpha_{1:k}^{(j)},Z_{1:k}^{(j)}, X_t)$$ 
			以下省去当前时间步$t$。对于n=1,...,m
			
	$$\textbf{m}_{j}^{n-1} = concat(\{{Z^{(j)}}_i^{n-1}\}_{i=1}^k)$$
	$$\tilde{\textbf{m}}_{j}^{n-1} = concat(\{\alpha_i^{(j)}{Z^{(j)}}_i^{n-1}\}_{i=1}^k) $$
			
$$\tilde{\mathbf{h}}_{j}^{n-1}=\left[\mathbf{m}_{j}^{n-1}\circ \mathbf{h}_{t}^{n-1}\right]$$
$$\hat{\mathbf{h}}_{j}^{n-1}=\left[\tilde{\mathbf{m}}_{j}^{n-1} \circ \mathbf{h}_{t}^{n-1}\right]$$
$$\mathbf{q}_{j}^{n}, \mathbf{k}_{j}^{n}, \mathbf{v}_{j}^{n}=\mathbf{h}_{j}^{n-1} \mathbf{W}_{q}^{n \top}, \tilde{\mathbf{h}}_{j}^{n-1} {\mathbf{W}_{k, E}^{n}}^{\top}, \hat{\mathbf{h}}_{j}^{n-1} \mathbf{W}_{v}^{n \top}$$
$$\mathbf{A}_{j, i}^{n}={\mathbf{q}_{j}^{n}}^{\top} \mathbf{k}_{i, j}^{n}+{\mathbf{q}_{j}^{n}}^{\top} \mathbf{W}_{k, R}^{n} \mathbf{R}_{i-j}+u^{\top} \mathbf{k}_{i, j}+v^{\top} \mathbf{W}_{k, R}^{n} \mathbf{R}_{i-j}$$
$$\mathbf{a}_{j}^{n}=\operatorname{Masked-Softmax}\left(\mathbf{A}_{j}^{n}\right) \mathbf{v}_{j}^{n}$$
$$\mathbf{o}^{n}=\text { LayerNorm (Linear }\left(\mathbf{a}^{n}\right)+\mathbf{h}^{n-1} )$$
$$\mathbf{h}^{n}=\text { Positionwise-Feed-Forward }\left(\mathbf{o}^{n}\right)$$


	\end{itemize}
\section{更新Memory}
	Memory的大小为N个Key-Value对,每个Value为一个片段$X_i$的所有层的表示。将上一步生成的$h_t^{1:m}$存入Memory:
	\begin{itemize}
		\item 将第一个Key-Value对删除,其余的向前递补,最后一个Key-Value对为空,将$h_t^{1:m}$填入Memory中的最后一个空位
	\end{itemize}
	对于填入的Key-Value对$h_t$,需要更新其Key,用最顶层表示更新Key:
	$$Key = summary(h_t^n)$$
	\begin{itemize}
		\item \textbf{3.1} \textbf{直接使用原向量} $summary(h_t^n) = Identical(flat(h_t^n))$
		\item \textbf{3.2} 线性变换 $summary(h_t^n) = Linear(flat(h_t^n))$
	\end{itemize}
%----------------------------------------------------------------------------------------


%----------------------------------------------------------------------------------------

\end{document}
