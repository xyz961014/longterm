%%%%%%%%%%%%%%%%%%%%%%%%%%%%%%%%%%%%%%%%%
% Journal Article
% LaTeX Template
% Version 1.4 (15/5/16)
%
% This template has been downloaded from:
% http://www.LaTeXTemplates.com
%
% Original author:
% Frits Wenneker (http://www.howtotex.com) with extensive modifications by
% Vel (vel@LaTeXTemplates.com)
%
% License:
% CC BY-NC-SA 3.0 (http://creativecommons.org/licenses/by-nc-sa/3.0/)
%
%%%%%%%%%%%%%%%%%%%%%%%%%%%%%%%%%%%%%%%%%

%----------------------------------------------------------------------------------------
%	PACKAGES AND OTHER DOCUMENT CONFIGURATIONS
%----------------------------------------------------------------------------------------

\documentclass{article}

\usepackage{xeCJK}
\usepackage{CJK}         % CJK 中文支持
\usepackage{fancyhdr}
\usepackage{ctex}
\usepackage{amsmath,amsfonts,amssymb,graphicx}    % EPS 图片支持
\usepackage{subfigure}   % 使用子图形
\usepackage{indentfirst} % 中文段落首行缩进
\usepackage{bm}          % 公式中的粗体字符(用命令\boldsymbol)
\usepackage{multicol}    % 正文双栏
\usepackage[sc]{mathpazo} % Use the Palatino font
\usepackage[T1]{fontenc} % Use 8-bit encoding that has 256 glyphs
\linespread{1.05} % Line spacing - Palatino needs more space between lines
\usepackage{microtype} % Slightly tweak font spacing for aesthetics

\usepackage[english]{babel} % Language hyphenation and typographical rules

\usepackage[hmarginratio=1:1,top=32mm,columnsep=20pt]{geometry} % Document margins
\usepackage[hang, small,labelfont=bf,up,textfont=it,up]{caption} % Custom captions under/above floats in tables or figures
\usepackage{booktabs} % Horizontal rules in tables



\usepackage{enumitem} % Customized lists
\setlist[itemize]{noitemsep} % Make itemize lists more compact



\usepackage{fancyhdr} % Headers and footers
\pagestyle{fancy} % All pages have headers and footers
\fancyhead{} % Blank out the default header
\fancyfoot{} % Blank out the default footer
\fancyhead[C]{Cache-based Recurrent Transformer Network} % Custom header text
\fancyfoot[RO,LE]{\thepage} % Custom footer text

\usepackage{titling} % Customizing the title section

\usepackage{hyperref} % For hyperlinks in the PDF

%----------------------------------------------------------------------------------------
%	TITLE SECTION
%----------------------------------------------------------------------------------------

\setlength{\droptitle}{-4\baselineskip} % Move the title up

\pretitle{\begin{center}\Huge\bfseries} % Article title formatting
\posttitle{\end{center}} % Article title closing formatting
\title{Cache-based Recurrent Transformer Network} % Article title
\author{%
% Your institution address
%\and % Uncomment if 2 authors are required, duplicate these 4 lines if more
%\textsc{Jane Smith}\thanks{Corresponding author} \\[1ex] % Second author's name
%\normalsize University of Utah \\ % Second author's institution
%\normalsize \href{mailto:jane@smith.com}{jane@smith.com} % Second author's email address
}
\date{} % Leave empty to omit a date

%----------------------------------------------------------------------------------------

\begin{document}

% Print the title
\maketitle

%----------------------------------------------------------------------------------------
%	ARTICLE CONTENTS
\section*{概览}
	对于时间步t,输入片段(向量)为$X_t \in R^{L\times d}$,Memory为$\mathcal{M} = \{K_i, V_i\}_{i=1}^N$,网络的更新步骤大致分为
	\subsection*{查询}
		\begin{equation}
			\{\alpha_i, Z_i\}_k = Query(X_t, \mathcal{M})
		\end{equation}
		其中$Z_i$为需要进行回忆(以Transformer Memory的方式拼接进当前区域)的k个区域,$\alpha_i$为其对应的权重\\
		$\mathcal{M}$是Memory,存储N个Key-Value对,Value即为区域,对应的Key为该区域的一个意义向量。
	\subsection*{更新hidden state}
		\begin{equation}
			h_t^{1:m} = Transformer(\alpha_{1:k},Z_{1:k}, X_t)
		\end{equation}
	\subsection*{更新Memory}
		\begin{equation}
			\mathcal{M} = renew(h_t^{1:m}, \mathcal{M})
		\end{equation}
\section{查询}
		\begin{itemize}
			\item \textbf{1.1} standard
			$$\{\alpha_i, Z_i\}_k = topk(softmax(summary(X_t)\cdot Keys^T))$$ 其中$summary$函数与之后更新Memory时使用的相同

		\item \textbf{1.2} computefirst:$\{\alpha_i, Z_i\}_k = Query(summary(Transformer(X_t)) ,\mathcal{M})$
			$$\{\alpha_i, Z_i\}_k = topk(softmax(summary(Transformer(X_t))\cdot Keys^T))$$
			其中$summary$函数与之后更新Memory时使用的相同, $Transformer$共享模型参数
	\end{itemize}

\section{更新hidden state}
	\begin{itemize}
		\item \textbf{3.1} standard:采用Transformer-XL的方法
			$$h_t^{1:m} = \text{Transfromer-XL}(\alpha_{1:k}, Z_{1:k}, X_t)$$ 
			对于n=1,...,m
			
	$$\textbf{m}_t^{n-1} = concat(Z_{1:k}^{n-1})$$
	$$\tilde{\textbf{m}}_t^{n-1} = concat(\{\alpha_iZ_i^{n-1}\}_{i=1}^k) $$
			
$$\tilde{\mathbf{h}}_{t}^{n-1}=\left[\operatorname{SG}\left(\mathbf{m}_{t}^{n-1}\right) \circ \mathbf{h}_{t}^{n-1}\right]$$
$$\hat{\mathbf{h}}_{t}^{n-1}=\left[\operatorname{SG}\left(\tilde{\mathbf{m}}_{t}^{n-1}\right) \circ \mathbf{h}_{t}^{n-1}\right]$$
$$\mathbf{q}_{t}^{n}, \mathbf{k}_{t}^{n}, \mathbf{v}_{t}^{n}=\mathbf{h}_{t}^{n-1} \mathbf{W}_{q}^{n \top}, \tilde{\mathbf{h}}_{t}^{n-1} {\mathbf{W}_{k, E}^{n}}^{\top}, \hat{\mathbf{h}}_{t}^{n-1} \mathbf{W}_{v}^{n \top}$$
$$\mathbf{A}_{t, i, j}^{n}={\mathbf{q}_{t, i}^{n}}^{\top} \mathbf{k}_{t, j}^{n}+{\mathbf{q}_{t, i}^{n}}^{\top} \mathbf{W}_{k, R}^{n} \mathbf{R}_{i-j}+u^{\top} \mathbf{k}_{t, j}+v^{\top} \mathbf{W}_{k, R}^{n} \mathbf{R}_{i-j}$$
$$\mathbf{a}_{t}^{n}=\operatorname{Masked}-\operatorname{Softmax}\left(\mathbf{A}_{t}^{n}\right) \mathbf{v}_{t}^{n}$$
$$\mathbf{o}_{t}^{n}=\text { LayerNorm (Linear }\left(\mathbf{a}_{t}^{n}\right)+\mathbf{h}_{t}^{n-1} )$$
$$\mathbf{h}_{t}^{n}=\text { Positionwise-Feed-Forward }\left(\mathbf{o}_{t}^{n}\right)$$


	\end{itemize}
\section{更新Memory}
	Memory的大小为N个Key-Value对,每个Value为一个片段$X_i$的所有层的表示。将上一步生成的$h_t^{1:m}$存入Memory:
	\begin{itemize}
		\item 将第一个Key-Value对删除,其余的向前递补,最后一个Key-Value对为空,将$h_t^{1:m}$填入Memory中的第一个空位
	\end{itemize}
	对于填入$h^{(t)}$的Key-Value对,需要更新其Key,用最顶层表示更新Key:
	$$Key = summary(h_t^n)$$
	\begin{itemize}
		\item \textbf{4.1} $Key = ReLU(W_S\cdot h_t^n + b_S)$
		\item \textbf{4.2} $Key = MLP(h_t^n)$
			$$(N \times)\ \ \ \  x = ReLU(Wx + b)$$
		\item \textbf{4.3} $Key = BiLSTM(h_t^n)$
	\end{itemize}
%----------------------------------------------------------------------------------------


%----------------------------------------------------------------------------------------

\end{document}
