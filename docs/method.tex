%%%%%%%%%%%%%%%%%%%%%%%%%%%%%%%%%%%%%%%%%
% Journal Article
% LaTeX Template
% Version 1.4 (15/5/16)
%
% This template has been downloaded from:
% http://www.LaTeXTemplates.com
%
% Original author:
% Frits Wenneker (http://www.howtotex.com) with extensive modifications by
% Vel (vel@LaTeXTemplates.com)
%
% License:
% CC BY-NC-SA 3.0 (http://creativecommons.org/licenses/by-nc-sa/3.0/)
%
%%%%%%%%%%%%%%%%%%%%%%%%%%%%%%%%%%%%%%%%%

%----------------------------------------------------------------------------------------
%	PACKAGES AND OTHER DOCUMENT CONFIGURATIONS
%----------------------------------------------------------------------------------------

\documentclass{article}

\usepackage{xeCJK}
\usepackage{CJK}         % CJK 中文支持
\usepackage{fancyhdr}
\usepackage{ctex}
\usepackage{amsmath,amsfonts,amssymb,graphicx}    % EPS 图片支持
\usepackage{subfigure}   % 使用子图形
\usepackage{indentfirst} % 中文段落首行缩进
\usepackage{bm}          % 公式中的粗体字符(用命令\boldsymbol)
\usepackage{multicol}    % 正文双栏
\usepackage[sc]{mathpazo} % Use the Palatino font
\usepackage[T1]{fontenc} % Use 8-bit encoding that has 256 glyphs
\linespread{1.05} % Line spacing - Palatino needs more space between lines
\usepackage{microtype} % Slightly tweak font spacing for aesthetics

\usepackage[english]{babel} % Language hyphenation and typographical rules

\usepackage[hmarginratio=1:1,top=32mm,columnsep=20pt]{geometry} % Document margins
\usepackage[hang, small,labelfont=bf,up,textfont=it,up]{caption} % Custom captions under/above floats in tables or figures
\usepackage{booktabs} % Horizontal rules in tables



\usepackage{enumitem} % Customized lists
\setlist[itemize]{noitemsep} % Make itemize lists more compact



\usepackage{fancyhdr} % Headers and footers
\pagestyle{fancy} % All pages have headers and footers
\fancyhead{} % Blank out the default header
\fancyfoot{} % Blank out the default footer
\fancyhead[C]{Cache-based Recurrent Attention Network} % Custom header text
\fancyfoot[RO,LE]{\thepage} % Custom footer text

\usepackage{titling} % Customizing the title section

\usepackage{hyperref} % For hyperlinks in the PDF

%----------------------------------------------------------------------------------------
%	TITLE SECTION
%----------------------------------------------------------------------------------------

\setlength{\droptitle}{-4\baselineskip} % Move the title up

\pretitle{\begin{center}\Huge\bfseries} % Article title formatting
\posttitle{\end{center}} % Article title closing formatting
\title{Cache-based Recurrent Attention Network} % Article title
\author{%
% Your institution address
%\and % Uncomment if 2 authors are required, duplicate these 4 lines if more
%\textsc{Jane Smith}\thanks{Corresponding author} \\[1ex] % Second author's name
%\normalsize University of Utah \\ % Second author's institution
%\normalsize \href{mailto:jane@smith.com}{jane@smith.com} % Second author's email address
}
\date{} % Leave empty to omit a date

%----------------------------------------------------------------------------------------

\begin{document}

% Print the title
\maketitle

%----------------------------------------------------------------------------------------
%	ARTICLE CONTENTS
\section*{概览}
	对于时间步t,输入词(向量)为$X^{(t)}$,网络的更新步骤大致分为
	\subsection*{查询}
		\begin{equation}
			\{\alpha_i, Z_i\}_k = Query(X^{(t)}, \mathcal{M})
		\end{equation}
		其中$Z_i$为需要进行Attention的k个区域,$\alpha_i$为其对应的权重\\
		$\mathcal{M}$是Memory,存储N个Key-Value对,Value即为区域,对应的Key为该区域的一个意义向量。
	\subsection*{Attention}
		\begin{equation}
			Attention_i = Attn(X^{(t)}, Z_i)
		\end{equation}
		或
		\begin{equation}
			Attention_i = Attn(h^{(t-1)},X^{(t)}, Z_i)
		\end{equation}
		
		\begin{equation}
			y^{(t)} = \sum_{i=1}^k \alpha_i Attention_i
		\end{equation}
	\subsection*{更新hidden state}
		\begin{equation}
			h^{(t)} = update(X^{(t)}, y^{(t)})
		\end{equation}
	\subsection*{更新Memory}
		\begin{equation}
			\mathcal{M} = renew(h^{(t)}, \mathcal{M})
		\end{equation}
\section{查询}
		\begin{itemize}
			\item \textbf{1.1} standard
			$$\{\alpha_i, Z_i\}_k = topk(softmax(X^{(t)}\cdot Keys^T))$$

		\item \textbf{1.2} 可能的变化:$\{\alpha_i, Z_i\}_k = Query(h^{(t-1)}, X^{(t)}, \mathcal{M})$
			$$\{\alpha_i, Z_i\}_k = topk(softmax((W_l\cdot concat(X^{(t)}, h^{(t-1)}) + b_l)\cdot Keys^T))$$
	\end{itemize}
\section{Attention}
	\begin{itemize}
		\item \textbf{2.1} $Attention_i = Attn(X^{(t)}, Z_i)$
			$$Q = ReLU(W_QX^{(t)} + b_Q)$$
			$$K = ReLU(W_KX^{(t)} + b_K)$$
			$$Attn(X^{(t)}, Z_i) = Attn(Q, K, Z_i)  = \sum_j softmax(Q\cdot K^T)_jZ_{ij}$$
			$$y^{(t)} = \sum_{i=1}^k \alpha_i Attention_i$$
		\item \textbf{2.2} $Attention_i = Attn(h^{(t-1)},X^{(t)}, Z_i)$
			$$Q = ReLU(W_Q\cdot concat(X^{(t)}, h^{(t-1)}) + b_Q)$$
			$$K = ReLU(W_KX^{(t)} + b_K)$$
			$$Attn(X^{(t)}, Z_i) = Attn(Q, K, Z_i)  = \sum_j softmax(Q\cdot K^T)_jZ_{ij}$$
			$$y^{(t)} = \sum_{i=1}^k \alpha_i Attention_i$$
	\end{itemize}
\section{更新hidden state}
	\begin{itemize}
		\item \textbf{3.1} standard
			$$h^{(t)} = ReLU(W_H\cdot concat(X^{(t)}, y^{(t)}) + b_H)$$
		\item \textbf{3.2} gated
			$$ r^{(t)} = \sigma(W_r \cdot concat(X^{(t)}, y^{(t)}) + b_r) $$
			$$ z^{(t)} = \sigma(W_z \cdot concat(X^{(t)}, y^{(t)}) + b_z) $$
			$$ n^{(t)} = tanh(r^{(t)} \cdot (W_nX^{(t)} + b_n) + W_i \cdot y^{(t)} + b_i) $$
			$$ h^{(t)} = (1 - z^{(t)})\cdot n^{(t)} + z^{(t)} \cdot y^{(t)}$$
	\end{itemize}
\section{更新Memory}
	Memory的大小为N个Key-Value对,每个Value为L个$h$的序列。分成两种情况来将上一步生成的$h^{(t)}$存入Memory:
	\begin{itemize}
		\item[1] Memory不满,直接将$h^{(t)}$填入Memory中的第一个空位
		\item[2] Memory满,将第一个Key-Value对删除,其余的向前递补,最后一个Key-Value对为空,成为第1种情况
	\end{itemize}
	对于填入$h^{(t)}$的Key-Value对,需要更新其Key:
	\begin{itemize}
		\item \textbf{4.1} $Key = ReLU(W_S\cdot Value + b_S)$
		\item \textbf{4.2} $Key = MLP(Value)$
			$$(N \times)\ \ \ \  x = ReLU(Wx + b)$$
		\item \textbf{4.3} $Key = BiLSTM(Value)$
		\item \textbf{4.3} $Key = Transformer(Value)$
	\end{itemize}
%----------------------------------------------------------------------------------------


%----------------------------------------------------------------------------------------

\end{document}
